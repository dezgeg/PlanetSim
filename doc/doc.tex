\documentclass[a4paper]{article}
\usepackage[utf8]{inputenc}
\usepackage[T1]{fontenc}
\usepackage{amsmath}
\usepackage[english]{babel}
\usepackage{hyperref}

\title{Scientific Computing IIa - Planetary Motion Simulator}
\author{Tuomas Tynkkynen}
\date{\today}

\begin{document}
 \maketitle
 \section{Description}
 This program simulates the motion of arbitrary number of celestial objects using the Velocity Verlet\cite{VV} numerical integration method.
  \subsection{Compiling}
  Run \texttt{make} to compile and link the program.
  The main executable program \texttt{planets} should then be created.

  To use the included animation script, both ImageMagick and Octave has to be installed as well.
 \section{Usage}
  \subsection{Running}
  \begin{center}
  Usage: \texttt{planets \emph{infile} [\emph{outfile}]}\\
  \end{center}
  The planets program reads the initial parameters for the simulation from the file \emph{infile} and writes the results to \emph{outfile}.
  If \emph{outfile} is omitted, the results are written to standard output instead.
  \subsection{File formats}
   The first line of the input file should be of the following format:
   \begin{center}$n$ $L$ $\Delta t$ $N_\omega$\end{center}
   where $n$ is the number of bodies; $L$ the duration of the simulation in seconds; $\Delta t$ the timestep of the simulation;
   $N_\omega$ the number of iteration steps written to the output file
   The following $n$ lines should contain
   \begin{center}$m_i$ $s_{x_i}$ $s_{y_i}$ $s_{z_i}$ $v_{x_i}$ $v_{y_i}$ $v_{z_i}$\end{center}
   where $m_i$ is the mass of the body $i$; $s_{x_i}$, $s_{y_i}$, $s_{z_i}$ the position of the body; 
   $v_{x_i}$ $v_{y_i}$ $v_{z_i}$ the velocity of the body

   The corresponding output file will contain $N_\omega$ lines of the form
   \begin{center} $n$ $t$ $m_1$ $s_{x_1}$ $s_{y_1}$ $s_{z_1}$ $m_2$ $s_{x_2}$ $s_{y_2}$ $s_{z_2}$ $\cdots$ $m_n$ $s_{x_n}$ $s_{y_n}$ $s_{z_n}$\end{center}
   where $n$ is the number of bodies; $t$ the time from the start of simulation; $m_i$; the mass of body $i$;
   $s_{x_i}$ $s_{y_i}$ $s_{z_i}$ the position of body $i$.
   \subsection{Creating an picture or animation of the results}
   An utility script, \texttt{animate.sh} can be used to create an animation of the simulation results.
   To create an animation, run \texttt{./animate.sh \emph{input-file} [\emph{-n}]}.
   In addition to the output file, this will generate a GIF animation and a static PNG image of the simulation in the \texttt{output/} folder.
   Use the \texttt{-n} parameter to skip creating the animation, since it usually takes a while.

   For example, running \texttt{./animate.sh testfiles/earthsun.in} creates the files \texttt{output/earthsun.out},
   \texttt{output/earthsun.gif}, \texttt{output/earthsun.png}.


\begin{thebibliography}{9}
\bibitem{VV} Wikipedia: Verlet integration -- \url{http://en.wikipedia.org/wiki/Verlet_integration#Velocity_Verlet}
\end{thebibliography}

\end{document}

